%----------------------------------------------------------------------------------------
%	PACKAGES AND OTHER DOCUMENT CONFIGURATIONS
%----------------------------------------------------------------------------------------

\documentclass[DIV=calc, paper=a4, fontsize=11pt, twocolumn]{scrartcl}	 % A4 paper and 11pt font size

\usepackage{lipsum} % Used for inserting dummy 'Lorem ipsum' text into the template
\usepackage[english]{babel} % English language/hyphenation
\usepackage[protrusion=true,expansion=true]{microtype} % Better typography
\usepackage{amsmath,amsfonts,amsthm} % Math packages
\usepackage[svgnames]{xcolor} % Enabling colors by their 'svgnames'
\usepackage[hang, small,labelfont=bf,up,textfont=it,up]{caption} % Custom captions under/above floats in tables or figures
\usepackage{booktabs} % Horizontal rules in tables
\usepackage{fix-cm}	 % Custom font sizes - used for the initial letter in the document

\usepackage{sectsty} % Enables custom section titles
\allsectionsfont{\usefont{OT1}{phv}{b}{n}} % Change the font of all section commands

\usepackage{fancyhdr} % Needed to define custom headers/footers
\pagestyle{fancy} % Enables the custom headers/footers
\usepackage{lastpage} % Used to determine the number of pages in the document (for "Page X of Total")

\renewcommand{\headrulewidth}{0.0pt} % No header rule
\renewcommand{\footrulewidth}{0.0pt} % Thin footer rule

\usepackage{lettrine} % Package to accentuate the first letter of the text
\newcommand{\initial}[1]{ % Defines the command and style for the first letter
\lettrine[lines=2,lhang=0.1,nindent=0em]{
\color{DarkGoldenrod}
{\textsf{#1}}}{}}

%----------------------------------------------------------------------------------------
%	TITLE SECTION
%----------------------------------------------------------------------------------------

\usepackage{titling} % Allows custom title configuration

\newcommand{\HorRule}{\color{DarkGoldenrod} \rule{\linewidth}{1pt}} % Defines the gold horizontal rule around the title

\pretitle{\vspace{-2pt} \begin{flushleft} \HorRule \fontsize{30}{30} \usefont{OT1}{phv}{b}{n} \color{DarkRed} \selectfont} % Horizontal rule before the title

\title{Hunchkin} % Your article title

\posttitle{\par\end{flushleft}\vskip 0.5em} % Whitespace under the title

\preauthor{\begin{flushleft}\large \lineskip 0.5em \usefont{OT1}{phv}{b}{sl} \color{DarkRed}} % Author font configuration

\author{A recommendation engine for hotels} % Your name

\postauthor{\footnotesize \usefont{OT1}{phv}{m}{sl} \color{Black} % Configuration for the institution name
% Your institution

\par\end{flushleft}\HorRule} % Horizontal rule after the title

\date{} % Add a date here if you would like one to appear underneath the title block

%----------------------------------------------------------------------------------------

\begin{document}

\maketitle % Print the title

\thispagestyle{fancy} % Enabling the custom headers/footers for the first page 
\section*{}

\initial{What it is }
Leisure travelers spend countless hours researching hotel features and reviews going back and forth between sites like Expedia, Kayak and Tripadvisor. When you travel alone for business, the decision variables are simple - location, wifi, dining etc. but family vacations can easily get complicated to research - activities for you and your partner, stuff for kids to do. Sometimes you can just go with a chain that you know but in many parts of the world, you really don't have that many chain hotels. The decision about which vacation hotel you pick is an important one because it can make or break a vacation. But it is not easy researching out all the information that you need to make this decision. It is further complicated by the huge volumes of reviews for popular hotels and also the increasing questionable quality of these hotels
\par
We have created a product that does the following:
\begin{itemize}
\item Asks you about a hotel that you liked in the past (could be from any location) during your hotel search along with dates of travel and destination
\item Gives you 5 recommendations which are similar to the hotel in features, amenities, dining, location etc.
\item Provides you with the ability to make detailed comparisons between these 5 hotels
\item You can ask your friends on Facebook or Google+ about opinions and recommendations from these 5 hotels
\end{itemize}

\subsection*{}

\initial{How it works }
Think of it as a Pandora for hotels. We have created a genome that represents various aspects of a hotel - its features, the location, the services it provides, the dining and several others and we collect this information from various sources. We have a unique way of finding a "genome distance" between hotels and we try to find hotels with the smallest distance (most similar to) from the hotel you say you liked. 

\subsection*{}

\initial{Revenue }
Hunchkin will initially rely primarily on affiliations with Online Travel Agencies (Expedia, Orbitz, Travelocity etc.) for referred sales commissions. Eventually as we establish ourselves as a key Hotel search platform, we will add advertising revenues too.


\subsection*{}

\initial{Competition }
Our competition will mainly be existing meta-search engines such as Kayak, Hipmunk etc. However, none of them have the recommendation capabilities that we are building. 


\end{document}
